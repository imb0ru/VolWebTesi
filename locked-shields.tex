\chapter{NATO CCDCOE Locked Shields}

L'esercitazione Locked Shields \cite{ccdcoe2025} rappresenta il principale banco di prova internazionale per le capacità di cyber defense, dove le competenze teoriche si confrontano con la complessità operativa della sicurezza informatica moderna. Il presente capitolo analizza il contesto distintivo di questa esercitazione, con particolare attenzione al ruolo dell'analisi forense digitale e ai requisiti emersi dall'edizione 2025 che hanno motivato lo sviluppo delle espansioni di VolWeb oggetto di questa tesi.

\section{Descrizione dell'Esercitazione}

\subsection{Storia e obiettivi di Locked Shields}

Locked Shields costituisce la più estesa e articolata esercitazione di cyber defense a livello globale, organizzata annualmente dal NATO Cooperative Cyber Defence Centre of Excellence (CCDCOE) con sede a Tallinn, Estonia. Concepita nel 2010 come risposta diretta agli attacchi cibernetici del 2007 contro l'Estonia, l'esercitazione ha conosciuto una crescita esponenziale: da evento regionale con limitata partecipazione si è trasformata in un'esercitazione di portata globale che coinvolge oltre 40 nazioni e più di 4000 specialisti del settore.

L'obiettivo principale di Locked Shields consiste nel fornire un ambiente simulato altamente realistico in cui team internazionali congiunti possano verificare e perfezionare le proprie competenze tecniche e procedurali in scenari di conflitto cibernetico su vasta scala. Diversamente dalle tradizionali competizioni Capture The Flag (CTF), che tendono a concentrarsi su sfide tecniche isolate, Locked Shields propone una simulazione integrata di warfare cibernetico, incorporando dimensioni tecniche, legali, di comunicazione strategica e cooperazione internazionale.

L'esercitazione si sviluppa attorno a uno scenario geopolitico fittizio ma verosimile, generalmente incentrato su una nazione immaginaria, bersaglio di una campagna di attacchi cibernetici coordinati. I team partecipanti assumono il ruolo di difensori delle infrastrutture critiche nazionali - reti elettriche, sistemi idrici, telecomunicazioni, servizi finanziari e apparati militari - con il duplice obiettivo di respingere gli attacchi e garantire la continuità operativa dei servizi essenziali. Tale approccio multidimensionale rispecchia fedelmente la natura della cyber warfare contemporanea, caratterizzata non da incidenti isolati ma da campagne orchestrate finalizzate alla destabilizzazione sistemica.

L'evoluzione di Locked Shields ha seguito parallelamente la trasformazione del panorama delle minacce cibernetiche. Se le prime edizioni si focalizzavano prevalentemente su attacchi Distributed Denial of Service (DDoS) e defacement di siti web, le iterazioni più recenti hanno introdotto scenari di complessità crescente: Advanced Persistent Threats (APT), supply chain attacks, campagne ransomware coordinate e compromissioni di sistemi di controllo industriale (SCADA/ICS). L'edizione 2025 ha segnato un ulteriore salto qualitativo, introducendo per la prima volta scenari basati su attacchi potenziati dall'intelligenza artificiale e operazioni di disinformazione mediante deepfake, anticipando le minacce emergenti nel dominio cibernetico.

\subsection{Struttura della competizione}

Locked Shields si articola in tre giorni di operazioni intensive, preceduti da settimane di preparazione mirata. L'esercitazione bilancia realismo operativo e obiettivi formativi attraverso fasi distinte.

\subsubsection{Fase preparatoria}
Durante le settimane precedenti l'esercitazione, i Blue Team accedono limitatamente all'infrastruttura virtualizzata per familiarizzare con i sistemi, identificare vulnerabilità e implementare misure di hardening. L'ambiente simulato comprende reti aziendali con Active Directory, sistemi SCADA/ICS per infrastrutture critiche, portali di servizi pubblici e infrastrutture militari C4ISR su reti 5G.

I team, rappresentando il Rapid Reaction Team nazionale, si organizzano in unità specializzate: network defense, amministrazione sistemi, sicurezza applicativa, gestione SIEM, analisi forense, consulenza legale e comunicazione strategica.

\subsubsection{Fase esecutiva}
I tre giorni operativi vedono i Blue Team fronteggiare attacchi progressivi del Red Team, esperti provenienti dal CCDCOE e nazioni partner. Gli attacchi seguono una narrativa APT realistica, dalla ricognizione alla compromissione sistemica.

Il Green Team gestisce l'infrastruttura e introduce complessità aggiuntive come guasti hardware o richieste operative urgenti. Il White Team arbitra e valuta le prestazioni secondo metriche predefinite.

Oltre alla difesa tecnica, i team producono deliverable critici: situation report, comunicati stampa, analisi legali e raccomandazioni strategiche, riflettendo la gestione completa di una crisi cibernetica.

\section{Ruolo del DFIR in Locked Shields}

\subsection{Importanza dell'analisi forense nelle CTF}

A differenza delle tradizionali competizioni CTF che privilegiano l'exploitation di vulnerabilità o la risoluzione di puzzle crittografici, Locked Shields attribuisce all'analisi forense digitale un ruolo centrale e distintivo. Tale scelta riflette la realtà operativa del mondo reale, dove la capacità di comprendere tempestivamente la natura, le modalità e le finalità di un attacco risulta spesso più determinante della prevenzione assoluta.

Nel contesto di Locked Shields 2025, l'attività forense si è configurata come una CTF di tipo Jeopardy quasi completamente autonoma rispetto alla rete di gioco principale di attacco e difesa. Questa struttura innovativa ha permesso ai team forensi di operare in un ambiente dedicato, affrontando 15 challenge specificamente progettate per testare diverse competenze investigative: 12 challenge di analisi post-mortem su dump di memoria e immagini disco, 2 challenge di analisi live su sistemi in esecuzione, e 1 challenge collaborativa da svolgere congiuntamente con i team Legal Advisor (LegAd) e Cyber Intelligence per la redazione di un report tecnico-legale integrato.

L'analisi forense in questo contesto ha svolto un ruolo multiplo e determinante per il successo complessivo del Blue Team. Ha permesso, innanzitutto, di attribuire le attività malevole e di comprendere le intenzioni degli attaccanti simulati: identificare chi fossero e quali obiettivi strategici stessero perseguendo ha dato modo ai team di anticipare le mosse successive, ottimizzando così l’allocazione delle risorse difensive. Lo studio dei pattern comportamentali e delle TTP adottate ha reso possibile distinguere tra operazioni di spionaggio, di sabotaggio o semplicemente di disruption, consentendo di modulare le contromisure in modo più efficace.

Un altro aspetto fondamentale è stata la rapida determinazione dell’ambito di compromissione. Comprendere con precisione quanto fosse estesa la violazione ha rappresentato un fattore cruciale per attuare decisioni di contenimento tempestive e mirate. La capacità di riconoscere la differenza tra un incidente isolato e una compromissione sistemica ha influito direttamente sulle strategie di risposta adottate.

Parallelamente, l’analisi dettagliata del malware e delle tecniche impiegate ha fornito preziose indicazioni sulle capacità offensive degli avversari simulati. Questa intelligence ha reso possibile l’adozione di contromisure calibrate e proporzionate alla minaccia. Non meno importante, soprattutto in scenari che prevedevano una possibile evoluzione verso risposte legali o diplomatiche, è stata la produzione di evidenze forensi adeguate agli standard probatori. Questo ha richiesto una particolare attenzione alla metodologia e alla documentazione, garantendo la solidità delle prove raccolte.

Il contesto altamente vincolato nel tempo di Locked Shields ha trasformato la tradizionale attività di DFIR — solitamente un processo metodico e sequenziale — in una vera e propria capacità operativa dinamica, fondata su decisioni rapide prese spesso sulla base di informazioni incomplete. Ciò ha imposto ai team l’adozione di strategie di triage aggressive e un ricorso intensivo all'automazione e a strumenti di supporto, con l’obiettivo di massimizzare l’efficienza analitica e mantenere un vantaggio tattico rispetto alle minacce.

\subsection{Metriche di valutazione}

Il sistema di valutazione adottato per il track forense in Locked Shields 2025 ha riflesso la multidimensionalità delle competenze richieste nel DFIR moderno, andando oltre la semplice correttezza tecnica per abbracciare aspetti processuali e strategici.

\paragraph{Metriche tecniche}
Le metriche tecniche hanno misurato l'efficacia nell'identificazione e analisi degli artefatti forensi:
\begin{itemize}
    \item Accuratezza nell'identificazione di Indicators of Compromise (IOC)
    \item Completezza nella ricostruzione della catena di eventi
    \item Precisione nell'attribuzione tecnica basata su TTP analysis
    \item Velocità di completamento delle challenge mantenendo standard qualitativi
\end{itemize}

\paragraph{Metriche processuali}
L'aderenza alle best practice forensi è stata valutata attraverso:
\begin{itemize}
    \item Documentazione della metodologia applicata e della chain of custody
    \item Riproducibilità delle analisi effettuate
    \item Gestione appropriata delle evidenze digitali
    \item Qualità della documentazione prodotta
\end{itemize}

\paragraph{Metriche collaborative}
La challenge integrata ha introdotto metriche specifiche per la collaborazione interdisciplinare:
\begin{itemize}
    \item Integrazione efficace tra componenti tecniche e legali del report
    \item Coerenza tra findings forensi e valutazioni di intelligence
    \item Chiarezza comunicativa per audience non tecniche
    \item Tempestività nella produzione del report integrato
\end{itemize}

Il sistema di scoring ha deliberatamente enfatizzato non solo la correttezza tecnica ma anche la capacità di contestualizzare i findings forensi nel più ampio scenario operativo, riflettendo l'evoluzione del ruolo del DFIR da funzione puramente tecnica a componente strategica della cyber defense.

\section{Requisiti specifici per Locked Shields 2025}

\subsection{Analisi delle esigenze identificate}

L'esperienza maturata durante Locked Shields 2025 ha evidenziato requisiti operativi specifici che gli strumenti di analisi forense tradizionali faticano a soddisfare nel contesto di esercitazioni ad alta intensità e, per estensione, in scenari operativi reali caratterizzati da pressione temporale estrema.

\paragraph{Velocità di analisi e scalabilità}
Il formato Jeopardy del track forense, con 15 challenge da completare in tempo limitato, ha posto l'accento sulla necessità di strumenti capaci di processare rapidamente grandi volumi di dati mantenendo accuratezza analitica. La capacità di analizzare molteplici dump di memoria di circa 20 GB ciascuno in tempi stringenti è diventata requisito fondamentale, richiedendo elaborazione parallela di multiple evidenze per massimizzare il throughput analitico. Questo implica la necessità di meccanismi di triage automatico che identificano rapidamente artefatti ad alta priorità e sistemi di caching intelligente per ottimizzare analisi iterative, evitando la ripetizione di operazioni computazionalmente intensive.

\paragraph{Pattern matching e threat hunting avanzati}
La complessità del malware presentato nelle challenge ha evidenziato come le semplici signature statiche siano ormai inadeguate. Il supporto per regole YARA complesse con logiche condizionali sofisticate è emerso come requisito critico, accompagnato dalla necessità di identificare varianti e mutazioni di malware noti attraverso matching fuzzy e analisi comportamentale. La correlazione automatica di pattern attraverso evidenze multiple e la generazione assistita di nuove regole basata su campioni identificati rappresentano capacità essenziali per mantenere il passo con l'evoluzione delle minacce.

\paragraph{Integrazione e correlazione di evidenze eterogenee}
La necessità di correlare informazioni provenienti da dump di memoria, immagini disco e analisi live ha sottolineato l'importanza di piattaforme veramente integrate. Un formato dati unificato che permetta l'aggregazione di risultati da fonti diverse, timeline correlate che integrino eventi da molteplici sorgenti, e capacità di cross-reference automatico tra artefatti correlati sono diventati requisiti imprescindibili. Le visualizzazioni unificate devono presentare una vista olistica dell'incidente, permettendo agli analisti di comprendere rapidamente relazioni complesse tra evidenze diverse.

\paragraph{Collaborazione e knowledge sharing}
La challenge collaborativa tecnico-legale ha evidenziato come il DFIR moderno richieda stretta cooperazione tra specialisti di domini diversi. Workspace condivisi per analisi collaborative in tempo reale, meccanismi strutturati di annotazione e tagging per facilitare la comunicazione interdisciplinare, template integrati per la produzione di report che soddisfino requisiti tecnici e legali, e integration points con piattaforme di cyber threat intelligence sono emersi come requisiti fondamentali per supportare questa nuova modalità operativa.

\subsection{Gap tra strumenti esistenti e requisiti}

L'analisi comparativa tra i requisiti identificati e le capacità degli strumenti esistenti ha rivelato gap significativi che limitano l'efficacia operativa nel contesto di Locked Shields e scenari simili.

\paragraph{Volatility Framework: completezza analitica vs. efficienza operativa}
Nonostante Volatility rimanga il riferimento per la completezza delle capacità analitiche, le sue limitazioni in contesti time-critical sono emerse con chiarezza. L'architettura monolitica e l'elaborazione sequenziale si sono rivelate inadeguate per l'analisi rapida di dump multipli. L'interfaccia a riga di comando, per quanto potente per utenti esperti, rallenta significativamente il processo di triage iniziale quando ogni minuto conta. L'output testuale non strutturato richiede post-processing manuale intensivo per identificare informazioni rilevanti, mentre l'integrazione limitata con YARA impedisce ricerche complesse basate su pattern comportamentali. L'assenza di funzionalità collaborative native rappresenta un ulteriore ostacolo in contesti dove il lavoro di team è essenziale.

\paragraph{Soluzioni commerciali: user experience vs. flessibilità}
Gli strumenti commerciali come Magnet AXIOM e Mandiant Redline, pur offrendo interfacce più intuitive, presentano limitazioni critiche in contesti competitivi. I modelli di licensing restrittivi limitano l'accesso simultaneo necessario per team numerosi o di natura non prettamente enterprise, mentre le capacità di customizzazione limitate impediscono l'implementazione rapida di nuove tecniche analitiche richieste da scenari in evoluzione. La natura proprietaria di questi strumenti preclude estensioni o modifiche per requisiti specifici, riducendone l'utilità in contesti che richiedono adattabilità estrema.

\paragraph{Frammentazione dell'ecosistema}
L'assenza di standard condivisi e l'incompatibilità tra strumenti creano inefficienze sistemiche che si amplificano sotto pressione temporale. La necessità di utilizzare tool multipli per coprire l'intero spettro di requisiti introduce overhead significativo nel trasferimento di dati e contesto. La duplicazione di effort analitico dovuta a incompatibilità di formati, la difficoltà nel mantenere una vista unificata dell'investigazione e la complessità nella gestione della chain of custody attraverso tool eterogenei rappresentano ostacoli operativi significativi che possono compromettere l'efficacia complessiva dell'analisi.

\subsection{Opportunità di miglioramento}

L'identificazione dei gap ha delineato opportunità concrete per l'evoluzione degli strumenti di analisi forense, con particolare riferimento all'integrazione di capacità avanzate in piattaforme unificate come VolWeb.

\paragraph{Architettura modulare e scalabile}
L'adozione di un'architettura basata su microservizi emerge come risposta naturale alle sfide identificate. La scalabilità orizzontale permette di gestire i carichi di lavoro variabili tipici delle esercitazioni, mentre l'isolamento dei componenti garantisce resilienza e manutenibilità. La possibilità di ottimizzare indipendentemente componenti critici per le performance, combinata con la facilità di integrazione di nuove funzionalità senza impatti sistemici, crea una piattaforma evolutiva capace di adattarsi a requisiti in continua evoluzione.

\paragraph{Integrazione nativa di capacità di pattern matching}
L'incorporazione profonda di YARA e tecnologie similari rappresenta un'opportunità trasformativa per l'efficacia investigativa. Un editor integrato per creazione e testing di regole con validazione in tempo reale può drammaticamente accelerare lo sviluppo di nuove detection. Ottimizzazioni specifiche per lo scanning di memoria, gestione efficiente di dataset di grandi dimensioni, e capacità di correlazione cross-evidenza per identificare pattern distribuiti possono moltiplicare l'efficacia analitica. L'applicazione di tecniche di machine learning per suggerire regole basate su nuovi campioni rappresenta la frontiera futura del pattern matching automatizzato.

\paragraph{Interfacce collaborative e knowledge management}
Il supporto nativo per workflows collaborativi può trasformare radicalmente l'efficacia dei team forensi. Workspace condivisi con sincronizzazione real-time permettono a specialisti diversi di lavorare simultaneamente sulla stessa evidenza, mentre template di report integrati assicurano che i risultati siano comunicati efficacemente a stakeholder tecnici, legali ed executive. L'integrazione con piattaforme di Threat Intelligence come MISP e OpenCTI, combinata con audit trail completi per requisiti di compliance, crea un ecosistema integrato che supporta l'intero ciclo di vita dell'investigazione.

\paragraph{Automazione context-aware}
L'implementazione di automazione intelligente rappresenta la chiave per gestire la complessità crescente mantenendo tempi di risposta rapidi. Pipeline di pre-processing che identificano e prioritizzano evidenze critiche possono ridurre drasticamente il tempo necessario per il triage iniziale. Classificatori basati su machine learning per la categorizzazione automatica di processi e comportamenti, combinati con la generazione automatica di IOC con confidence scoring, permettono agli analisti di concentrarsi sugli aspetti più critici dell'investigazione. L'orchestrazione di analisi complesse attraverso playbook personalizzabili standardizza le best practice mantenendo la flessibilità necessaria per scenari non standard.

Queste opportunità di miglioramento non rappresentano mere ottimizzazioni incrementali, ma trasformazioni fondamentali necessarie per colmare il divario tra le capacità attuali e i requisiti operativi emersi da contesti ad alta intensità come Locked Shields. La loro implementazione in piattaforme integrate come VolWeb può significativamente elevare le capacità di risposta della comunità DFIR di fronte a minacce sempre più sofisticate.

Tra le diverse opportunità identificate, l'integrazione del supporto YARA in VolWeb è emersa come priorità immediata per diverse ragioni convergenti. La centralità del pattern matching nelle challenge forensi di Locked Shields 2025, combinata con la maturità tecnologica di YARA, ha reso questa integrazione il punto di partenza naturale per l'evoluzione della piattaforma. Inoltre, il supporto YARA rappresenta un enabler fondamentale per molte delle altre capacità identificate: l'automazione del threat hunting, la generazione di IOC, e la correlazione cross-evidenza dipendono tutte da robuste capacità di pattern matching. Pertanto, il lavoro di tesi si è concentrato prioritariamente sull'implementazione di un sistema completo di gestione, validazione ed esecuzione di regole YARA all'interno di VolWeb, ponendo le basi per future espansioni che potranno costruire su questa fondazione.

I requisiti emersi da Locked Shields 2025 hanno delineato una roadmap chiara per l'evoluzione di VolWeb. Il prossimo capitolo presenta l'analisi dettagliata, la progettazione e l'implementazione delle espansioni YARA, mostrando come queste trasformino VolWeb da strumento di visualizzazione a piattaforma completa per il threat hunting avanzato.