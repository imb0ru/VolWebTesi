\chapter{Codice Sorgente e Documentazione Tecnica}

Il codice sorgente completo delle espansioni di VolWeb presentate in questa tesi è disponibile pubblicamente su GitHub. Il repository contiene l'implementazione completa del modulo YARA, le modifiche architetturali discusse, e la documentazione dettagliata per l'installazione e l'utilizzo.

\section{Repository GitHub}

Il progetto è ospitato all'indirizzo:
\begin{center}
\url{https://github.com/imb0ru/volweb}
\end{center}

Il repository segue una struttura modulare che separa chiaramente le componenti backend e frontend, facilitando la comprensione e la manutenzione del codice. La sezione Wiki fornisce istruzioni dettagliate per il setup dell'ambiente di sviluppo e il deployment in produzione. Il file README include una panoramica delle funzionalità principali.

\section{Contributi e Licenza}

Il progetto è rilasciato sotto GNU General Public License versione 3 (GPLv3), permettendo l’uso, la modifica e la distribuzione libera del software, a condizione che eventuali versioni modificate siano anch’esse rilasciate sotto la stessa licenza. I contributi sono benvenuti attraverso pull request su GitHub.

La scelta di una licenza open source riflette la filosofia di democratizzazione dell’accesso agli strumenti di memory forensics che ha guidato lo sviluppo di VolWeb fin dalla sua concezione originale.